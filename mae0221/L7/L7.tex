\documentclass[9.5pt,reqno,a4paper]{amsart}

%SetFonts

%%%% Pacotes adicionais para enumeração de items e alinhamento de equações
%%%%
\usepackage{enumitem}
\usepackage{amsmath}
\usepackage{mathtools}

%%%% Permite syntax highlight
%%%%
\usepackage{minted}
\usepackage{xcolor}
\definecolor{softgray}{gray}{0.96}
\setminted{
linenos,
framesep=10px,
baselinestretch=1.2,
bgcolor=softgray
}

%%%% Escrevendo em português
%%%% 
\usepackage[utf8]{inputenc}

%%%% Configurações para output correto de pdf
%%%%
\usepackage[T1]{fontenc}
\usepackage{ae}
\usepackage{aecompl}
\usepackage{amssymb}

%%%% Layout de página
%%%%
\usepackage{fullpage}
\usepackage{setspace}
\usepackage{bbm}

\begin{document}
\parindent=0pt

\title[MAE0221]
{MAE0221 Probabilidade I}\vspace{3\jot}%\\\vspace{1\jot}

% \footskip=10pt

\maketitle
\thispagestyle{empty} 
\pagestyle{plain}
\onehalfspace

\textbf{Nome: Ygor Sad Machado}\hfill
\textbf{Número USP: 8910368}\null

\medskip
\textbf{Lista 7}\hfill
\textbf{Data: 27/05/2020}\null

\noindent\rule{\textwidth}{0.4pt}

\bigskip
Considere inicialmente o seguinte código de \textit{setup}. Note que ele possui a definição de algumas funções que serão utilizadas em todas as simulações a seguir.

\bigskip
\begin{minted}{python}
from random import random, seed

# Fixa a semente do gerador de números aleatórios para manter
# a consistência nos resultados da simulação
seed(1)

# Calcula a distância Manhattan entre dois pontos p1 e p2
def manhattan_distance(p1, p2):
    return sum(abs(a-b) for a, b in zip(p1, p2))

# Gera um ponto aleatório num grid quadrado de lado L
def rand_coord(l):
    return [random() * l, random() * l]

# Realiza o experimento um determinado número de vezes para
# um valor de L e um número de ambulâncias fixados
def experiment(num, l, ambulances):
    sum_distance = 0

    for i in range(int(num)):
        accident = rand_coord(l)

        # Gera uma lista contendo o número de ambulâncias recebido
        coord_ambulances = [rand_coord(l) for j in range(ambulances)]

        # Soma somente a menor distância entre alguma ambulância e
        # o local do acidente
        sum_distance += min(
            [manhattan_distance(accident, amb) for amb in coord_ambulances]
        )

    return sum_distance / num
\end{minted}

\newpage
\noindent\rule{\textwidth}{0.4pt}
\textbf{Item a.}

Neste item estamos interessados em descobrir a esperança para a distância entre dois pontos usando $L=1$ como parâmetro. Para isso foi usado o seguinte código:

\begin{minted}{python}
def item_a():
    print("A. ")

    for n in [1e5, 1e6, 1e7]:
        expectation = experiment(num=n, l=1, ambulances=1)
        print("{it:.1e} iterations => {e}".format(it=n, e=expectation))

    print("\n")
\end{minted}

A execução dessa função gera como saída:

\begin{minted}{python}
1.0e+05 iterations => 0.666830572178
1.0e+06 iterations => 0.666817365745
1.0e+07 iterations => 0.666774043139
\end{minted}

Isso sugere uma convergência no valor da esperança, de modo que $\mathbb{E} \approx 0.66666...$ parece ser uma boa aproximação para ela.

\bigskip
\noindent\rule{\textwidth}{0.4pt}
\textbf{Item b.}

Já neste item, estamos interessados em descobrir uma relação entre o valor $L$ e a esperança $\mathbb{E}$ para a distância, estando o número de ambulâncias fixado em $n=1$. Para isso usamos o seguinte código:

\begin{minted}{python}
def item_b():
    print("B. ")

    for length in range(2,8):
        expectation = experiment(num=1e6, l=length, ambulances=1)
        print("L={l} => {e}".format(l=length, e=expectation))

    print("\n")
\end{minted}

Cuja saída é:

\begin{minted}{python}
L=2 => 1.333307524
L=3 => 2.00006665411
L=4 => 2.66444358418
L=5 => 3.33203882442
L=6 => 3.99962416282
L=7 => 4.6727126035
\end{minted}

Note que todos os valores que obtivemos correspondem a dois terços do valor de seus respectivos $L$, o que nos sugere que a esperança pode ser aproximada por $\mathbb{E} = \frac{2L}{3}$.

\bigskip
\noindent\rule{\textwidth}{0.4pt}
\textbf{Item c.}

Já para estas simulações, vamos fixar o número de ambulâncias em $n=2$ e executar o código abaixo.

\begin{minted}{python}
def item_c():
    print("C. ")

    for n in [1e5, 1e6, 1e7]:
        expectation = experiment(num=1e6, l=1, ambulances=2)
        print("{it:.1e} iterations => {e}".format(it=n, e=expectation))

    print("\n")
\end{minted}

Para ele obtivemos a seguinte saída:

\begin{minted}{python}
1.0e+05 iterations => 0.487881704132
1.0e+06 iterations => 0.487956896953
1.0e+07 iterations => 0.487897822258
\end{minted}

Essa saída nos sugere que um boa aproximação para a esperança nessas condições é $\mathbb{E} \approx 0.488$.

\bigskip
\noindent\rule{\textwidth}{0.4pt}
\textbf{Item d.}

Por fim, neste item queremos descobrir o menor número de ambulâncias necessárias para que a esperança da distância entre o local do acidente e a ambulância mais próxima seja $\frac{L}{3}$. Para isso fixamos um valor inicial arbitrariamente alto para $\mathbb{E}$ e vamos incrementando o número de ambulâncias até que $\mathbb{E} \leq \frac{L}{3}$. Além disso, variamos $L$ e o número de experimentos de modo a ter um panorama para diferentes tamanhos de \textit{grid}. O código abaixo foi usado nessa simulação.

\begin{minted}{python}
def item_d():
    print("D.")

    for n in [1e5, 1e6, 1e7]:
        print("{it:.1e} iterations".format(it=n))

        for length in range(1,5):
            num_ambulances = 0
            expectation = 1e20

            while (expectation >= length / 3.0):
                num_ambulances += 1
                expectation = experiment(num=n, l=length, ambulances=num_ambulances)

            print(
                "L={l} | expectation={e} => {amb} ambulances".format(
                    l=length, e=expectation, amb=num_ambulances
                )
            )
        print("\n")
\end{minted}

A execução de dessa função tem como saída:

\begin{minted}{python}
1.0e+05 iterations
L=1 | expectation=0.31074490733 => 5 ambulances
L=2 | expectation=0.622679090939 => 5 ambulances
L=3 | expectation=0.934401355886 => 5 ambulances
L=4 | expectation=1.24493205269 => 5 ambulances

1.0e+06 iterations
L=1 | expectation=0.310965785717 => 5 ambulances
L=2 | expectation=0.621694269052 => 5 ambulances
L=3 | expectation=0.932674487682 => 5 ambulances
L=4 | expectation=1.24288063528 => 5 ambulances

1.0e+07 iterations
L=1 | expectation=0.311061783846 => 5 ambulances
L=2 | expectation=0.621796002463 => 5 ambulances
L=3 | expectation=0.932712301752 => 5 ambulances
L=4 | expectation=1.24334908597 => 5 ambulances
\end{minted}

Observe que, independentemente do número de experimentos realizados ou do tamanho de $L$, o resultado dos experimentos aponta sempre para 5 ambulâncias. Parece razoável então afirmar que o número mínimo de ambulâncias é constante e sempre igual a 5.

\end{document}