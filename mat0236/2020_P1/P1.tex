\documentclass[12.5pt,reqno,a4paper]{amsart}
%SetFonts

%SetFonts

%%%% Pacotes adicionais para enumeração de items e alinhamento de equações
%%%%
\usepackage{enumitem}
\usepackage{amsmath}
\usepackage{mathtools}


%%%% Escrevendo em português
%%%% 
\usepackage[utf8]{inputenc}

%%%% Configurações para output correto de pdf
%%%%

\usepackage[T1]{fontenc}
\usepackage{ae}
\usepackage{aecompl}
\usepackage{amssymb}

%%%% Layout de página
%%%%
\usepackage{fullpage}
\usepackage{setspace}
\usepackage{bbm}

%%%% Comandos personalizados
%%%%
\def\Assin#1{\noindent\textit{Assinatura}\strut\\%:\\
\framebox[\textwidth]{\phantom{\vrule height#1}}}
\newcommand\numberthis{\addtocounter{equation}{1}\tag{\theequation}}
\DeclarePairedDelimiter{\ceil}{\lceil}{\rceil}
\DeclarePairedDelimiter{\floor}{\lfloor}{\rfloor}

%%% Valor absoluto
\DeclarePairedDelimiter\abs{\lvert}{\rvert}%
\DeclarePairedDelimiter\norm{\lVert}{\rVert}%

% Swap the definition of \abs* and \norm*, so that \abs
% and \norm resizes the size of the brackets, and the 
% starred version does not.
\makeatletter
\let\oldabs\abs
\def\abs{\@ifstar{\oldabs}{\oldabs*}}
%
\let\oldnorm\norm
\def\norm{\@ifstar{\oldnorm}{\oldnorm*}}
\makeatother

\begin{document}

\parindent=0pt

\title[MAT0236 - P1]
{\textit{MAT0236 Funções Diferenciáveis e Séries}\\\vspace{3\jot}}%\\\vspace{1\jot}

\footskip=28pt

\maketitle
\thispagestyle{empty} 
\pagestyle{plain}
\onehalfspace

\textbf{Nome: Ygor Sad Machado}\hfill
\textbf{Número USP: 8910368}\null

\medskip
\textbf{Prova 1 - Solução}\hfill
\textbf{Data: 10/04/2020}\null

\noindent\rule{\textwidth}{0.4pt}

\medskip
\textbf{Lista 2 - Ex. 7.4}
\medbreak
Queremos determinar a convergência da seguinte série:

\begin{equation*}
    \sum_{n = 1}^{\infty} \frac{2^n}{(n!)^\lambda}, \lambda > 0
\end{equation*}

\bigskip
A presença de um fatorial no denominador torna conveniente aplicar o teste da razão. Para tanto, vamos definir

\begin{equation}
    a_n = \frac{2^n}{(n!)^\lambda} \numberthis \label{7-4_original_series}
\end{equation}

\bigskip
Em posse de $a_n$, aplicamos o seguinte limite:

\begin{align*}
    L &= \lim_{n \to \infty} \abs{\frac{a_{n+1}}{a_n}} \\
      &= \lim_{n \to \infty} \abs{\frac{2^{n+1}}{(n+1)!^\lambda} \cdot \frac{(n!)^\lambda}{2^n}} \\
	  & = \lim_{n \to \infty} \abs{\frac{2^{n+1}}{2^n} \cdot \frac{(n!)^\lambda}{(n+1)!^\lambda}}\\
	  &= \lim_{n \to \infty} \abs{\frac{2}{(n+1)^\lambda}} \numberthis \label{7-4_final_abs}
\end{align*}

\bigskip
Observe que \eqref{7-4_final_abs} é sempre maior que zero, pois $n$ vai ao infinito a partir de 1 e $\lambda > 0$. Logo, sem prejuízo, podemos tirar o módulo e continuar o cálculo do limite.

\begin{align*}
    L &= \lim_{n \to \infty} \frac{2}{(n+1)^\lambda} \\
      &= 2 \cdot \lim_{n \to \infty} \frac{1}{(n+1)^\lambda} \\
      &= 2 \cdot 0 \\
      &= 0
\end{align*}

\bigskip
Ora, segundo o teste da razão, se $L < 1$, então a série é absolutamente convergente. Logo, como chegamos a $L = 0$, podemos deduzir que \eqref{7-4_original_series} converge absolutamente, e portanto converge sempre.
\qed\null

\newpage
\textbf{Lista 2 - Ex. 8.1}
\medbreak

Queremos determinar a convergência da seguinte série

\begin{equation*}
    \sum_{n = 1}^{\infty} {(-1)^n \frac{1}{\sqrt{n}}} \numberthis \label{8-1_original_series}
\end{equation*}

\bigskip
Para isso, comecemos avaliando sua convergência absoluta. 

\begin{align*}
    \sum_{n = 1}^{\infty} \abs{(-1)^n \frac{1}{\sqrt{n}}} =
    \sum_{n = 1}^{\infty} \abs{\frac{(-1)^n}{\sqrt{n}}} =
    \sum_{n = 1}^{\infty} \frac{1}{\sqrt{n}} \\ \numberthis \label{8-1_abs_failed}
\end{align*}

\bigskip
É fácil notar que \eqref{8-1_abs_failed} é divergente, pois ela é uma p-série com $p = \frac{1}{2}$. Já sabemos, portanto, que a série não converge absolutamente, precisamos então verificar sua convergência condicional. Para isso, vamos aplicar o teste da série alternada. Pelos critérios desse teste, \eqref{8-1_original_series} será convergente se, definindo $a_n = \frac{1}{\sqrt{n}}$:

\bigskip
$(i)$. $\lim_{n \to \infty} a_n = 0$

\smallskip
$(ii)$. $a_n$ for decrescente, ie, $a_{n+1} \leq a_n$ para $n$ suficientemente grandes. 

\bigskip
\bigskip
Começando pelo critério $(i)$, é fácil notar que ele vale para o nosso cenário. Como $f(x) = \sqrt{x}$ é uma função crescente, sua discretização também o é. Logo, se fizermos $\lim_{n \to \infty} \frac{1}{\sqrt{n}}$, à medida que $n$ cresce, o denominador cresce junto, de modo que $\frac{1}{\sqrt{n}}$ descresce, aproximando-se cada vez mais de zero. Assim $\lim_{n \to \infty} \frac{1}{\sqrt{n}} = 0$.

\bigskip
Verificar o critério $(ii)$ significa afirmar que a razão entre dois valores consecutivos de $a_n$ é sempre inferior ou igual a 1, a partir de um dado $n$. 

\begin{equation}
    \left(\frac{a_{n+1}}{a_n}\right) \leq 1, n \geq N, \text{ para algum } N \geq 1
\end{equation}

\medskip
Expandindo:

\begin{align*}
    \left(\frac{a_{n+1}}{a_n}\right) =
    \left(\frac{1}{\sqrt{n+1}} \cdot \frac{\sqrt{n}}{1}\right) =
    \left(\frac{\sqrt{n}}{\sqrt{n+1}}\right) \numberthis \label{8-1_final_abs}
\end{align*}

\bigskip
\bigskip
Fixemos então $N = 1$. Como já foi exposto no critério $(i)$, $\sqrt{n}$ é crescente, o que significa que $\sqrt{n + 1} \geq \sqrt{n}$. Daí podemos deduzir então que, para qualquer $n > N$:

\begin{equation}
    \left(\frac{\sqrt{n}}{\sqrt{n+1}}\right) \leq 1
\end{equation}

\bigskip
\bigskip
Isso mostra que $a_n$ é uma sequência decrescente, o que implica que o critério $ii$ também é verdadeiro. Logo, a série \eqref{8-1_original_series} converge absolutamente pelo teste da série alternada.
\qed\null

\newpage
\textbf{Lista 2 - Ex. 10.3}
\medbreak

Considere a série abaixo.

\begin{equation*}
    \sum_{n = 2}^{\infty} {(-1)^{n + 1} \frac{1}{n ^ \text{ln x}}} \numberthis \label{10-3_original_series}
\end{equation*}

\bigskip
Estamos interessados em saber para quais $x \in \mathbb{R}$ ela converge. Para tanto, comecemos verificando sua convergência absoluta.

\begin{align*}
    \sum_{n = 2}^{\infty} \abs{(-1)^{n + 1} \cdot \frac{1}{n^\text{ln x}}}
    &= \sum_{n = 2}^{\infty} \abs{\frac{(-1)^{n + 1}}{n^\text{ln x}}}\\
    &= \sum_{n = 2}^{\infty} \frac{1}{n^\text{ln x}} \numberthis \label{10-3_abs_conv}
\end{align*}

\bigskip
Para que \eqref{10-3_original_series} possua convergência absoluta – e portanto convergência –, \eqref{10-3_abs_conv} precisa convergir. Observe que esta última se assemelha a uma p-série com $p = \ln{x}$. Ora, mas sabemos que p-séries convergem se $p > 1$. Logo, para que \eqref{10-3_abs_conv} convirja, $\ln{x} > 1$, isto é, $x > e$.

\bigskip
Sintetizando, \eqref{10-3_original_series} converge para todo $x > e$, $x \in \mathbb{R}$.
\qed\null

\endgroup
\end{document}